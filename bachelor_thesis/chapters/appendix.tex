\csvstyle{Activities}{
tabular = |c|l|l|,
table head = \hline \textbf{Case ID} & \textbf{Activity} & \textbf{Event Type}\\\hline\hline,
late after line = \\\hline,
head to column names,
}
\csvstyle{SingleColumn}{
tabular = |l|,
table head = \hline \textbf{#1} \\\hline\hline,
late after line = \\\hline,
head to column names,
}
\section{Appendix}\label{sec:apx}
% \addcontentsline{toc}{section}{Appendix}
\addtocontents{toc}{\protect\setcounter{tocdepth}{-1}}
\subsection{Patient Journeys}\label{apx:pjs}
\subsubsection{Patient Journey 1}\label{apx:pj1}
I started experiencing symptoms of Covid-19 on July 15, 2021. As a 51-year-old teacher from Germany, I knew the importance of taking immediate action. I isolated myself at home and informed my family about the situation. Despite the financial difficulties we were facing, my main concern was the health and safety of my loved ones.\\
In the next few days, my symptoms worsened, and I decided to consult a doctor. I reached out to my family physician, who advised me to get tested for Covid-19. I made an appointment at a local testing center and underwent the necessary tests. The results confirmed my infection, and the doctor recommended home quarantine and symptomatic treatment.\\
During my isolation period, I focused on resting and following the doctor's advice. My loving wife took care of me and ensured that I had everything I needed. My children, understanding the seriousness of the situation, helped around the house and supported me emotionally.\\
As the weeks went by, I slowly recovered from the infection, thanks to the care and support of my family. I returned to work as a teacher, taking necessary precautions to protect both myself and my students.\\
In the following months, the Covid-19 situation began to improve, and vaccination campaigns were initiated. I made the decision to get vaccinated to further protect myself and those around me. I consulted with my doctor about the vaccine and its benefits for me as an individual. After careful consideration, I received my first dose of the vaccine in the month of September 2021.\\
Following the initial dose, I experienced mild side effects such as fatigue and soreness at the injection site. However, I remained optimistic and continued with my daily life while adhering to the safety guidelines.\\
In the weeks that followed, I received my second dose of the vaccine, completing the vaccination process. I felt a sense of relief and hope for the future, knowing that I had taken the necessary steps to protect myself and contribute to the collective effort of ending the pandemic.\\
Throughout this whole journey, I am grateful for the love and support I received from my family and the guidance of the healthcare professionals. Their care and expertise played a significant role in my recovery and ability to resume my work as a teacher.
\subsubsection{Patient Journey 2}\label{apx:pj2}
I started experiencing symptoms of Covid-19 on June 15, 2021. As a doctor, I immediately recognized the signs and knew I needed to take action. I self-isolated in a separate room of my house to protect my family and prevent any potential spread of the virus. The next day, I contacted my primary care physician and informed them about my symptoms.\\
Over the course of the next few days, my symptoms progressed, and I started experiencing high fever, cough, and fatigue. I diligently followed the guidelines provided by healthcare authorities, taking over-the-counter medication to manage my symptoms and staying hydrated. I continued to stay isolated at home, avoiding close contact with others.\\
Concerned about the severity of my symptoms, I reached out to a colleague who specialized in infectious diseases for a telemedicine consultation. They provided valuable advice on managing my symptoms and recommended regularly monitoring my oxygen levels using a pulse oximeter.\\
As the week progressed, my symptoms gradually improved. However, I remained cautious and continued to self-isolate to ensure the virus had fully run its course. After two weeks of isolation, I finally started feeling better and was able to resume my normal activities.\\
In terms of vaccination, after recovering from Covid-19, I decided to get vaccinated to provide additional protection against future infections. I received the first dose of the vaccine a few months later in September 2021, following the recommended schedule. I experienced mild side effects, including soreness at the injection site and fatigue, but they subsided within a few days.\\
I received the second dose of the vaccine in October 2021, completing the vaccination process. I followed up with my primary care physician to ensure I had developed a strong immune response to the virus. They reassured me that the vaccine would significantly reduce the likelihood of reinfection and the severity of any future infections.\\
Overall, my experience with Covid-19 highlighted the importance of following public health guidelines, seeking medical advice, and taking necessary precautions to protect both myself and my loved ones.
\subsubsection{Patient Journey 3}\label{apx:pj3}
As a 35-year-old doctor constantly on the go, juggling between patients and hospital rounds, I found myself infected with Covid-19 on August 15, 2022. It started with mild symptoms like fatigue and a slight cough, but quickly progressed to include body aches and a high fever. Concerned about the potential impact on my patients, I immediately isolated myself at home.\\
Over the next few days, my symptoms worsened, and I experienced difficulty breathing. Knowing the seriousness of the situation, I reached out to a fellow doctor for a virtual consultation. They advised me to monitor my oxygen levels regularly and prescribed medications to alleviate my symptoms. With their guidance, I managed to stabilize my condition and avoid hospitalization.\\
As the days passed, I realized the toll the virus was taking on my body. My energy levels were depleted, making it challenging for me to carry out my daily duties. Thankfully, my colleagues stepped in and offered to cover my hospital rounds, allowing me time to rest and recover.\\
In the following weeks, I focused on self-care, ensuring I followed a healthy diet, stayed hydrated, and got enough rest. I also relied on virtual support groups to connect with others who had experienced Covid-19, providing a sense of camaraderie during these difficult times.\\
With the recovery and discovery of vaccines, I made the decision to get vaccinated as soon as it became available to me. I received both doses of the vaccine within the recommended timeframe, which provided me with a sense of relief and protection against future infections.\\
Throughout this entire experience, I couldn't help but reflect on my single status. Being isolated during my illness made me long for companionship even more. As I approached my late thirties, I wondered if I would ever find love amidst my hectic schedule. However, I remained hopeful and determined to prioritize my health and well-being while keeping an open heart to whatever the future might hold.\\
In summary, my Covid-19 infection led me to take immediate action by isolating myself and seeking medical advice. I relied on virtual consultations, support groups, and the support of my colleagues to navigate through the challenging period of illness. Eventually, I seized the opportunity to get vaccinated, completing the recommended doses. The experience reinforced the importance of self-care and strengthened my resolve to find love despite the demands of my career.
\subsubsection{Patient Journey 4}\label{apx:pj4}
I had my first interaction with covid in early 2021. It all started with everyone getting back to school after the long break over the summer of 2020.\\
Because we were wearing masks all the time the immune system was really down. Idk how to say it, but every little cold hit harder than before. And that's why I didn't think much about it when I started coughing.But then it progressed to also fever but i never had typical problems like shortness of breath or all in all really heavy symptoms.I still went to school though as I had to prepare for my graduation, so I didn't test myself so i dont really know, if i really had it back then.However in the last weeks we had to test us every second day of school and some day in April i had a positiv test, so I had to go in quarantine and to get a negative pcr test like one week later at the local doctors'.\\
In this time I had no symptoms at all lol. I guess it was like october in 2021 when I finally got the first dose of vaccine against covid. The second was six or seven weeks later and someday in the first half of 22 i got the third round. I never understood antivaccers, but thats not my beer anyway...\\
I'm just happy that me and all people around me hadn't hard times with corona.
\subsubsection{Patient Journey 5}\label{apx:pj5} (Shortened from~\cite{malta_my_2020})\\
I'm a global health researcher working to address health and gender inequalities in the Global South. During my work in areas where Malaria or Dengue Fever are endemic, I always took extra precautions to avoid getting infected. I never anticipated that while living in a large, urban city from Canada I would be at higher risk… Until the COVID-19 pandemic.
During lockdown, like most working mothers, I became the major responsible for childcare and housework. To finish all my research related activities, I frequently worked until late at night. During the day I was juggling work, home and homeschooling… In mid May I started feeling weak and had more trouble breathing. As someone with an immunodeficiency disorder, I didn't pay too much attention. I though it was due to sleep deprivation and excessive working hours… But it was COVID-19. The symptoms worsened quickly and in a few days I was not able to get out of bed. Now I was under lockdown, unable to work or look after my kids, with stress piling up.\\\\
My physician considered the symptoms mild, recommending isolation and rest at home… I laughed: How does someone isolate and rest with three little kids at home and so much work to do? I was bedridden for three weeks, with difficulty breathing, headache, conjunctivitis, sore throat, aches and pain. I completely lost my appetite. During two months I could not taste or smell anything, hot or cold, sweet, salty, spice, nothing at all. My fatigue was debilitating. More than four months later, my symptoms have not gone away. My heart still races a few times a day - even while I am sitting at the computer and writing this piece. It is hard to concentrate for long periods. Imagine a scientist that cannot concentrate properly… That's me. What about the stress? It keeps piling up, with no light on the horizon.\\
I'm what has been identified as a ‘long-hauler’, those individuals who survived a COVID-19 infection but are enduring long-term symptoms. Fatigue is one of the most common symptom, while many report racing heartbeat, enduring achy joints and shortness of breath.\\\\
Like many long-haulers, my goal is to resume previous normal and productive life. However, I still experience a plethora of long-term symptoms, including extreme fatigue and brain fog. Many long-haulers fear unemployment, as they face incapacitating symptoms and/or lower productivity. The burden of having a possible long-term condition is very stressful. But when allied with the prospects of unemployment, and for many, loss of health insurance, it can be unbearable. I'm grateful for being alive, still employed and living in a country offering universal, publicly funded healthcare. Many don't have the same privilege.
\newpage
\subsection{Ground Truths}\label{apx:ground_truths}
\begin{table}[h]
    \tiny
    \captionsetup{justification=raggedright,singlelinecheck=false}
    \caption{Ground truth for Patient Journey 1~\ref{apx:pj1}}
    \csvreader[Activities]{bachelor_thesis/data/journey_comparison_1_ground_truth.csv}{event_type=\eventtype}{%
        1 & \activity & \eventtype
    }
    \label{tab:pj1-gt}
\end{table}
\begin{table}[h]
    \tiny
    \captionsetup{justification=raggedright,singlelinecheck=false}
    \caption{Ground truth for Patient Journey 2~\ref{apx:pj2}}
    \csvreader[Activities]{bachelor_thesis/data/journey_comparison_2_ground_truth.csv}{event_type=\eventtype}{%
        2 & \activity & \eventtype
    }
    \label{tab:pj2-gt}
\end{table}
\begin{table}[h]
    \tiny
    \captionsetup{justification=raggedright,singlelinecheck=false}
    \caption{Ground truth for Patient Journey 3~\ref{apx:pj3}}
    \csvreader[Activities]{bachelor_thesis/data/journey_comparison_3_ground_truth.csv}{event_type=\eventtype}{%
        3 & \activity & \eventtype
    }
    \label{tab:pj3-gt}
\end{table}
\begin{table}[h]
    \tiny
    \captionsetup{justification=raggedright,singlelinecheck=false}
    \caption{Ground truth for Patient Journey 4~\ref{apx:pj4}}
    \csvreader[Activities]{bachelor_thesis/data/journey_comparison_4_ground_truth.csv}{event_type=\eventtype}{%
        4 & \activity & \eventtype
    }
    \label{tab:pj4-gt}
\end{table}
\begin{table}[h]
    \tiny
    \captionsetup{justification=raggedright,singlelinecheck=false}
    \caption{Ground truth for Patient Journey 5~\ref{apx:pj5}}
    \csvreader[Activities]{bachelor_thesis/data/journey_comparison_5_ground_truth.csv}{event_type=\eventtype}{%
        5 & \activity & \eventtype
    }
    \label{tab:pj5-gt}
\end{table}
\addtocontents{toc}{\protect\setcounter{tocdepth}{2}}