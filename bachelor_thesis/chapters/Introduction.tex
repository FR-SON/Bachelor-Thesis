\section{Introduction}\label{sec:intro}
Over recent years, Process Mining has proven to be a powerful tool for the detection, analysis, and optimization of (business) processes~\cite{weske_business_2012}. Its operation is fundamentally reliant on event logs, which traditionally are generated automatically within Information Systems.\\
However, with the rise of large language models (LLMs), interest in processing unstructured data has increased considerably. Should we be able to analyse not only structured data from companies, healthcare systems, for example, but also non-standard reports from arbitrary individuals, it would open up entirely new possibilities. These could range from analysing conversations from social media platforms such as Reddit or Facebook, to specifically requested reports that do not adhere to a strict format, making them easier for people to produce.\\
Consequently, there exists a crucial need to convert unstructured data into event logs, without which Process Mining cannot take place. For instance, mamahealth is currently working on a project that utilizes natural language testimonials from individuals with chronic illnesses. The project views these accounts from a process perspective, with the ultimate goal of generating potentially life-altering insights.\\
The project TracEX, developed in collaboration with mamahealth, plays an important role in this research and will be further introduced in the following chapters.It provides an extraction framework that this thesis aims to improve by providing fine-tuned models. There are significant challenges in sourcing and preparing data for Process Mining, even when using traditional data sources~\cite{van_der_aalst_process_2016}. From a practical standpoint, data quality is paramount for successful Process Mining. Any missing or untrustworthy event data severely undermines the value of the results obtained. 
\begin{quote}
    \quotes{From a practical point of view data quality is of the utmost importance for the success of process mining. If event data is missing or cannot be trusted, then the results of process mining are less valuable.}~\cite{van_der_aalst_process_2016}    
\end{quote}
These problems persist when unstructured text is utilized as a data source. Furthermore, the data has to be structured, which in turn complicates matters. Non-standardized, non-automated data, written by laypeople without any claim for completeness or readability, are even more prone to the issues faced with traditional data sources.\\
In the past, attempts have been made to process this type of data, including in the realm of process modelling~\cite{friedrich_process_2011}. Aside from  human manual extraction,  deterministic Natural Language Processing (NLP) approaches have been used. Unfortunately, these attempts were quickly met with limitations, particularly in deriving temporal relationships between events.\\
Recently, there has been an explosive rise in the field of Generative AI and LLMs in general. Transformer models from OpenAI, for instance, have proven to be powerful tools in various applications due to their ability to comprehend human language and respond in kind. This suggests that these models could also process textual disease course descriptions (patient journeys).\\
However, the use of LLMs introduces additional challenges. They are non-deterministic, meaning the same input can yield different outputs. Furthermore, their operation mechanisms are not transparent and therefore not entirely understood. Making them carry out tasks exactly as desired requires significant trial and error. For event logs, not only the quality of the collected data is vital, but so is the formatting. The data must not only be extracted accurately and completely, but also be cast into the appropriate format (such as XES).
These and many other hurdles pose substantial difficulties in event log extraction, as described in~\cite{munoz-gama_process_2022}. Established Transformer models from, as those from OpenAI, struggle to overcome these issues. Therefore, research has been conducted to explore ways to fine-tune these general-purpose models for specific tasks or topics. This work investigates the extent to which fine-tuning could potentially solve these problems and create high-quality event logs. As stated by~\cite{latif_fine-tuning_2024}, 
\begin{quote}
\quotes{Fine-tuned GPT models are more suited to tasks like text completion, response evaluation, or open-ended queries because of their autoregressive nature, which excels in sequence formation.} 
\end{quote}

This thesis is structured as follows:\\
Chapter~\ref{sec:back} introduces the preliminaries followed by Chapter~\ref{related_work}, situating this thesis among other studies. Chapter~\ref{sec:fine} explores different approaches to fine-tuning and describes the compilation of training data. In Chapter~\ref{sec:eval}, the performance of the trained models is evaluated after introducing the metrics required to do so. The thesis concludes with Chapter~\ref{sec:conclusion}, summarizing findings, discussing the limitations and providing an outlook for future work.

\subsection{Background}\label{sec:back}
\subsubsection*{Event Logs}\label{sec:event-log}
Event Logs, as the name suggests, are collections of recorded events and their defining data, related to an observed process. They can be used as input for process mining and therefore need to adhere to a certain structure. Usually, they are represented as tables, where each row represents one event and each column contains the values of the events' attributes. The bare minimum for these attributes is \emph{case ID} and \emph{activity}, assuming an event is made from a series of activities. Each instance of the process, from which the events originate, is labelled with an ID, the case ID. All events related to the same case are referred to as a \emph{trace}. An event log can contain multiple traces. On top of the mandatory attributes, timestamps, activity categories and more process specific data can be recorded in an event log.~\cite{van_der_aalst_process_2016}

\subsubsection*{Patient Journeys}\label{sec:pj}
Patient journeys are natural language texts without any defined format or structure. They describe a person's course of disease from the patient's perspective, as it is usually written by the patient themself. They  are commonly found on the internet, notably on social media platforms such as X, Facebook, and Reddit. Characteristically, they encompass events spanning beyond a single hospital stay, offering a broader picture by compiling experiences from before the onset of the condition, during hospital visits, at home, during doctor's appointments, and so on. Often, these accounts are enriched with personal feelings, either explicitly stated or implicitly conveyed through the writing style. The lack of structure and absence of guidelines on what content must be included come with both advantages and disadvantages.

\subsubsection*{ChatGPT}\label{sec:chatgpt}
ChatGPT is an advanced conversational AI model developed by OpenAI. It signifies a considerable advancement in natural language processing (NLP). This powerful model is adept at understanding and generating human-like text, making it useful in a wide array of applications.\\
Fundamentally, a large language model like ChatGPT is a statistical tool, trained to predict the probability distribution of word sequences. By analysing vast datasets of text, the model discerns patterns, structures, and semantic relationships within the data, enabling it to produce coherent and contextually suitable text.\\  
ChatGPT is constructed on the Transformer architecture, which includes several key components. Notably, the attention mechanism plays a vital role. Within this framework, self-attention calculates attention scores for each word relative to others in a sequence, allowing the model to emphasize significant words during text generation. Multi-head attention further refines this capability by employing multiple attention heads to capture various types of relationships and dependencies.  Building on this foundation, GPT-3.5, an iteration of the model, enhances the scale and capabilities seen in its predecessor, GPT-3. Notably, GPT-3.5 inherits the expansive architecture of GPT-3, which has approximately 175 billion parameters. Parameters, essentially weights learned during the model's training, are crucial in determining its functionality and performance.  ChatGPT 3.5 Turbo harnesses extensive computational resources and large, diverse text corpora during its training.\\
This process can be broken down into several critical phases. Pre-training involves exposing the model to a vast corpus of text data in an unsupervised manner, allowing it to learn language structures and patterns. Following this, fine-tuning adjusts the model's performance on specific tasks using labelled data. The final phase, optimization and deployment, involves rigorous testing and refinement to ensure efficiency and reliability in real-world applications.\\
In operation, ChatGPT's mechanism starts with input processing, where user prompts are tokenized into a format the model can process. Tokens are subunits of text, such as words or subwords, that the model analyses. The generative process then involves these tokens being processed through numerous layers, ultimately generating probabilities for the next token in the sequence and decoding these generated tokens back into human-readable text.


\subsubsection*{Fine-Tuning}\label{sec:fine-tuning-def}
Fine-tuning Large Language Models (LLMs) means customizing their operation to align with the specific demands of a domain or task. Predominantly, a pre-trained LLM, such as a ChatGPT model, forms the foundation for this process, which primarily entails continuous training or iterative updates to the model's knowledge base. The objective here is to optimize the model's ability to manage domain or task-specific requests, which could relate to either knowledge or response behaviour, including response format.\\
There are multiple fine-tuning methodologies to boost the potential of LLMs, including supervised fine-tuning, unsupervised fine-tuning, and reinforcement learning methods. \emph{Supervised Fine-Tuning} (SFT) employs labelled input-output pairs, which implies that each instance in the training dataset is equipped with the intended output that the algorithm should independently generate. \emph{Unsupervised fine-tuning} uses unlabelled datasets, meaning there is no ground truth element. The data is processed by the model without specific instructions. This implies, the model uses methods like clustering and anomaly detection to find structure or patterns in the data. In \emph{reinforcement learning}, a model strives to optimize its performance in a specific task. It is an iterative approach, that includes multiple rounds of feedback and a reward, that  depends on the models' performance. With the goal to find the strategy that yields the greatest overall reward, the model continuously adapts and improves.

\paragraph{Training Data}
When using SFT, the model learns solely from examples. It is given a system-message, a user-message and an assistant-message. The system-message describes the behaviour the model should adopt and the task it should perform. The user-message contains the input the model should perform the task on. The assistant-message then contains the desired answer – the response the model should give when confronted with this task in the future.
In a way, one message set closely resembles one turn of a normal exchange with ChatGPT with the difference being, that the answer is also given to the model instead of the model calculating the answer itself.
Example:
\begin{lstlisting} [language=json, caption={Example for fine-tuning training data}, label={lst:fine-tuning-example}]   
{"messages":[
    {"role":"system",
     "content":"You are an expert in text categorization. Your job is to take a given activity label and to classify it into one of the following event types: 'Symptom Onset', 'Symptom Offset', 'Diagnosis', 'Doctor Visit', 'Treatment', 'Hospital Admission', 'Hospital Discharge', 'Medication', 'Lifestyle Change' and 'Feelings'. Please consider the capitalization."
    },
    {"role":"user", "content":"testing positive for Covid19"},
    {"role":"assistant","content":"Diagnosis"}]}
\end{lstlisting}
The training data is made up of ten to multiple hundred examples.

\subsubsection*{TracEX}\label{sec:tracex}
TracEX\footnote{https://github.com/FR-SON/TracEX} is a LLM-based data processing pipeline that aims to extract event logs from patient journeys. It takes a patient journey as an input and delivers an event log (in XES format) as an output. The extraction process involves many steps, some of which will be important later on. Hence, they are listed below:
\paragraph{Preprocessing} Patient Journeys are often unstructured, making them hard to use for information extraction. Events are not always described in chronological order, time related information is often missing or relative to each other (e.g. \quotes{two weeks later}) and language is not standardized (use of synonyms, paraphrases or slang). Extracting data from this reliably is virtually impossible. This is why preprocessing is important. The patient journey is moulded into a more structured format, in order to streamline following steps, while still retaining the continuous text format. This includes correcting spelling and grammar as well as identifying and transforming time related phrases into a date format (e.g. YYYY-MM-DD). Additionally, case information like the age of the author and the described condition are extracted.
\paragraph{Extracting Event Types} Given an activity label, the category of that activity is determined from a predefined list.
\paragraph{Labelling Activities} In contrast to event types, activity labels are not chosen from a predefined list of possible activities. Using the entire patient journey as context information and also considering the condition, all relevant events are determined. They are then each summarized into 6 words at most. This implies, that the exact phrasing of an activity can vary, each time the pipeline is executed. An example for a typical activity is \quotes{experiencing fever-like symptoms}.
(Terminology: Activity and event are used synonymously in this case, because the distinction is not needed. There is no hierarchical order to events in TracEX, meaning they do not contain multiple activities.)
\paragraph{Extracting Timestamps} Given an activity, its start and end date are determined. It is extracted if possible and extrapolated if not. Furthermore, the duration of the activity is calculated from the two timestamps.\\\\
To achieve the described outcome, a mix of deterministic scripts and requests to the ChatGPT-API are used. Among other techniques, prompt engineering is used to increase the quality of the extracted event logs.
Few-shot has proven to be especially suited for the described tasks. A request to the OpenAI API might contain a message list like this:
\begin{lstlisting}[language=json, caption={Few-shot prompt to categorize activities into event types}, label={lst:few-shot}]
{"messages":[
	{"role": "system",  
     "content": "You are an expert in text categorization and your job is to take a given activity label and to classify it into one of the following event types: 'Symptom Onset', 'Symptom Offset', 'Diagnosis', 'Doctor Visit', 'Treatment', 'Hospital Admission', 'Hospital Discharge', 'Medication', 'Lifestyle Change' and 'Feelings'. Please consider the capitalization.",  
	},  
    {"role": "user", "content": "visiting doctor's"},  
    {"role": "assistant", "content": "Doctors Visit"},  
    {"role": "user", "content": "testing positive for Covid19"},
    {"role": "assistant", "content": "Diagnosis"},  
    {"role": "user", "content": activity_label}  
]}
\end{lstlisting}
A prompt containing the task for the model and a list of examples to specify the models' behaviour is given, along with the activity label that is to be classified.
Please note that TracEX is an open-source tool. It may very well still be in development and functionality might be altered, removed or added in the future. The development status this thesis builds upon can be seen 
