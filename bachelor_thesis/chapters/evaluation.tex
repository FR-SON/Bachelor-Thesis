\section{Evaluation}\label{sec:eval}
In this chapter we will evaluate the presented approaches by using the trained models from Chapter~\ref{sec:multi-task-ft} and \ref{sec:single-task-ft} to extract event logs from patient journeys. In order to properly compare the fine-tuned models and the base model, we will first introduce custom metrics and a tool that will support the evaluation. Second, we will describe the evaluation process. Third, we will present our results for each model, each task and each of the task's metrics individually.\\
The evaluation is based on comparing the extracted event logs to a ground truth and determining the differences between them.

\subsection{Metrics}\label{sec:metrics}
In \ref{sec:multi-task-ft} we describe the process of fine-tuning a model to perform multiple related tasks. We evaluate each of these tasks individually.

\subsubsection{Activity Labelling Metrics}\label{sec:activity_metrics}
There already are established metrics in the process mining sector we can use to determine if an event log is of good quality in terms of activities. Therefore, we just briefly explain how they work.\\
It is worth noting, that determining if one activity from the evaluated event log and one from the ground truth are indeed the same activity is not as straight forward as one might think. As discussed in \ref{sec:back} the activity labels are not predetermined, instead they are chosen by the model that extracts them. This means two activity labels can be syntactically different but still describe the same activity. For the purpose of this evaluation, we define two activities as identical, if they are semantically very similar.
\paragraph{Missing Activity} An activity is missing, if it is featured in the ground truth, but not in the evaluated event log. 
\paragraph{Unexpected Activity} An activity is unexpected, if it is featured in the evaluated event log, but not in the ground truth.
\paragraph{Wrong Order} Two activities are in wrong order, if both of them are featured in the ground truth as well as the evaluated event log, but their order of appearance is interchanged. In our case, the order of activities is defined by their start timestamp. This means two activities $A_1 \text{ and } A_2$ are in wrong order, if they appear in the order $A_1,A_2$ in the ground truth and in order $A_2, A_1$ in the evaluated event log.\\\\
As this is a comparative analysis, the value we use from this metric is the percentage of errors in each of the mentioned categories – Missing Activity, Unexpected Activity and Wrong order. This way, the number of errors is relative to the number of activities in the respective patient journey.

\subsubsection{Event Type Classification Metric}\label{sec:eventtype_metric}
We use a set of event types so each pair of them is mutually exclusive. This means for every activity in the ground truth there is exactly one correct event type. This fact reduces this metric to two possible results per event type: \emph{True} and \emph{False.}\\
The value we use from this metric is the percentage of \verb|True| and \verb|False| values respectively. This way, the number of errors is relative to the number of activities in the respective patient journey.
Possible event types are: Symptom Onset, Symptom Offset, Doctor Visit. Hospital Admission, Hospital Discharge, Treatment, Medication, Lifestyle Change, Feelings.

\subsection{Timestamp Extraction Metrics}\label{sec:time_metrics}
Similar to the event type metric, a timestamp can be either extracted correctly or incorrectly, resulting in \verb|True| or \verb|False| when evaluated. In order to adhere to the commonly used standard in event logs, that is also required to produce valid XES files, the timestamps are in the format \verb|YYYY-MM-DD|. In many cases, including the sample patient journeys we use in the evaluation, patient journeys do not contain time information this precise. Most time specifications are relative and vague. For the purpose of this evaluation, we define a timestamp to extracted correctly, if the date is plausible in the context of the patient journey. An activity, that is described to have happened on \quotes{some day in April…}, any timestamp between 01. April and 30. April would be correct. This is unless, there is other time related information that further specify the timeframe in which an activity must have happened. Let $A_1$ be an activity that happens before activity $A_2$ and $t_2=2024-04-15$ be the timestamp for $A_2$. If $A_1$ is specified as \quotes{some day in April…}, only timestamps between 01. April and 14. April would be correct. The same goes for \quotes{as the weeks progressed}, \quotes{later that year} and similar phrases. As long as an extracted timestamp is correct relatively to all other information given in the patient journey and is a s precise as the text provides, it is deemed correct.
